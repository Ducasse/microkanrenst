
\documentclass[a4paper,12pt]{article}

\usepackage{inputenc}
\usepackage{euler}
\usepackage[T1]{fontenc}
\usepackage{libertine}
\usepackage{lipsum}
\usepackage{fancyvrb}
\usepackage{url}
\usepackage[english]{babel}
\usepackage{amsmath}
\usepackage{amsthm}
\usepackage{amssymb}
\usepackage{acronym}
\usepackage{hyperref}
\usepackage{tabu}
\usepackage{rotating}
\usepackage{mathdots}
\usepackage{minted}
\usepackage{units}
\usepackage{float}
\usepackage{bbold}
\usepackage[sort&compress,square,comma,authoryear]{natbib}

\fvset{fontsize=\normalsize}
\setmonofont[Scale=0.8]{Menlo}
\usemintedstyle{xcode}
\hypersetup{colorlinks=false, citecolor=Violet}

\newtheorem{theorem}{Theorem}
\newtheorem{lemma}[theorem]{Lemma}
\newtheorem{proposition}[theorem]{Proposition}
\newtheorem{corollary}[theorem]{Corollary}
\newtheorem{definition}[theorem]{Definition}
\newtheorem{remark}[theorem]{Remark}
\newtheorem{example}[theorem]{Example}

\newcommand{\stcode}[1]{\mintinline[fontsize=\footnotesize]{smalltalk}{#1}}

\author{Massimo Nocentini}
\title{Relational programming in Smalltalk}

\begin{document}

\maketitle

\begin{abstract}
This booklet is an extended description of $\mu$Kanren, a relational
interpreter formerly implemented in Scheme and ported in Smalltalk for the sake
of understanding and, of course, fun. It enjoys simplicity and elegance despite
the complex nature of logic systems; moreover, using an uniform but powerful
language such as Smalltalk it is possible to appreciate and benefit from both
properties.
\end{abstract}

\section*{Introduction}


This document extends the presentation of $\mu$Kanren shown at the last ESUG
conference \citep{ESUG2018} in two ways: (i)~it recaps existing literature 
on the subject and (ii)~it describes in more details our implementation coded
in Smalltalk.

It is a waste of time to repeat ourself on $\mu$Kanren's definition; on the
contrary, we believe interesting to look up and to watch over the many
implementations it has inspired since its seminal paper \citep{Hemann:muKanren}
and how programmers stretch the canonical implementation to take advantage of
their favorite programming language's features.

Certainly the book \citep{Friedman:Reasoned:Schemer} is the main reference
written in the same style of the other \emph{little books} of which Dan
Friedman coauthors them all; moreover, Byrd's dissertation \citep{Byrd:PhD}
shows advanced stuffs and techniques while the repository
\url{http://minikanren.org/} -- updated regularly by Byrd himself -- keeps
pointers to implementations, articles and presentations.  $\mu$Kanren is
\emph{purely functional} therefore any language that has the same power of the
$\lambda$-calculus is a candidate for implementing it, nonetheless its first
definition has been written in Scheme; in parallel, the implementation with
symbolic constraints has the same nature but uses macros and in some aspect it
requires more effort to port it to other languages that haven't macros.

Another implementation using the \emph{Clojure} dialect of Lisp is
\citep{clojure:core.logic} which targets the JVM and provides constraints and
nominal logic programming; on the other hand, \citep{sullivan:microkanrenhs}
is a clean and concise implementation using Haskell, while many others simpler
versions can be found, written for the sake of understanding and, ultimately,
for fun.

Finally, your humble authors believe that $\mu$Kanren is a sophisticated
\textit{educational machine} because it concerns and mixes together unification,
lazy streams, backtracking and optimization techniques; moreover, it allows
programmers to write their own version according to their personal taste and
style, ranging from implementations pretty imperative to ones built over
categories. 

Our effort to understand $\mu$Kanren started in \citep{Nocentini:kanrens} using
the \emph{Python} programming language; we take advantage of native
\Verb|generator|s objects and we provide a \emph{fair} enumeration strategy
that the original implementation hasn't; on the contrary, we experience
computational limits because of \texttt{RecursionError} exceptions. This
implementation is complete and parallels the one shown in the reference book,
accompanied by a unit test suite that provides answers to \emph{all} questions
proposed in the book and some new applications such as (i)~finding fixed points
of a set of inference rules and (ii)~enumerating classes of combinatorial
objects. Moreover, two \textit{impure} operators, as the \textit{cut}
(\Verb|!|) Prolog operator, had been coded too. However complete, the code is a
mixture of the object orientation at the surface and functional at the core both
supported by primitives of the underlying language, such as \Verb|yield|
expressions and magic methods; took note of this, we desire something more
uniform and cohesive.

The current work describes a Smalltalk implementation of the basic system with
the aim to learn and experiment with a pure object environment, such as Pharo
and Squeak. The work \citep{evan:smallkanren} is another Smalltalk
implementation of $\mu$Kanren that currently supports disequality constraints
and nominal logic; it is a more complete and complex version that we point it
out for users requiring those features -- hoping to be able to provide those
features by ourselves.

Starting small, the next example shows three different
approaches to define the addition of three \ct{SmallInteger} objects toward 
the introduction of the relational paradigm.
\begin{example}
In math a relation $P$ is usually characterized by a statement such as
\begin{displaymath}
\forall a,b,c.\,P(a,b,c) \leftrightarrow a + b = c \quad\text{from which}\quad
P(1,2,3) \quad\text{can be proved.}
\end{displaymath}
The same check can be expressed in Smalltalk using either the
\textit{imperative style}
\begin{minted}[fontsize=\footnotesize]{smalltalk}
a := 1.  b := 2.  c := a + b.  Object assert: [ c = 3 ]
\end{minted}
or the \textit{functional style}
\begin{minted}[fontsize=\footnotesize]{smalltalk}
Object assert: [ ([ :a :b | a + b ] value: 1 value: 2) = 3 ]
\end{minted}
or, finally, the \textit{declarative style}
\begin{minted}[fontsize=\footnotesize]{smalltalk}
Object assert: [ [ :a :b :c | a + b = c ] value: 1 value: 2 value: 3 ]
\end{minted}
which closely resembles the mathematical statement.
\end{example}


\tableofcontents

\section{Preparing the wires...}

\subsection{\texttt{Goal}s and \texttt{State}s}

A \textit{goal} is any object that responds to the selector \Verb|#onState:| so
that it receives a substitution and produces an ordered sequence of zero or
more substitutions. In turn, a substitution is an unordered collection of
values which are indexed by \textit{logic variables}, therefore a
\Verb|Dictionary| object fits perfectly; we wrap it in a \Verb|State| object
that also keeps record of an incremental numbering \Verb|birthdate| whose
meaning will be clarified later when we'll describe the \Verb|Fresh|
goal\footnote{put here the reference to its subsection}. An empty substitution
is created by
\begin{minted}[fontsize=\footnotesize]{smalltalk}
Dictionary new asState
\end{minted}
where
\begin{minted}[fontsize=\footnotesize]{smalltalk}
Dictionary>>#asState
    ^ State new
        birthdate: 0;
        substitution: self;
        yourself
\end{minted}
and it can be augmented by
\begin{minted}[fontsize=\footnotesize]{smalltalk}
State>>#at: aVar put: aValue
    | s |
    s := substitution copy.
    s "check that if a key already exists at aVar, then the values must be the same"
      at: aVar
      ifPresent: [:v | aValue = v ifFalse: [UnificationError signal]]
      ifAbsentPut: aValue.
    ^ self class new
        birthdate: birthdate;
        substitution: s;
        yourself
\end{minted}
mantaining a coherent protocol with respect to the one of \Verb|Dictionary|
objects for extension; moreover, \Verb|UnificationError| objects will be
explained in a later section when will talk about the unification process.

\subsection{\texttt{Chain}s and \texttt{Cons}es}

Because the sequence of substitutions may be infinite, we
represent it not as a \Verb|LinkedList| object but as a \Verb|Chain| object;
as usual, we find \Verb|BlockClosure| objects good candidates to represent (a possibly
infinite) computation. Some examples are in order,
\begin{minted}[fontsize=\footnotesize]{smalltalk}
ChainTest>>#collatz: o
    ^ o
        cons: [o even
                ifTrue: [self collatz: o / 2]
                ifFalse: [self collatz: 3 * o + 1]]

ChainTest>>#fibs
    ^ self fib: 0 fib: 1

ChainTest>>#fib: m fib: n
    ^ m cons: [self fib: n fib: m + n]

ChainTest>>#gcd: m gcd: n
    ^ m cons: [n isZero
                ifTrue: [Chain with: m]
                ifFalse: [self gcd: n gcd: m \\ n]]

ChainTest>>#nats
    ^ self ints: 0

ChainTest>>#ints: i
    ^ i cons: [self ints: i + 1]

ChainTest>>#zeroOneTwo
    ^ 0 cons: (1 cons: (2 cons: Chain bottom))
\end{minted}
where any \Verb|Object| object respond to the message
\begin{minted}[fontsize=\footnotesize]{smalltalk}
Object>>#cons: anObj
    ^ anObj consedObject: self
\end{minted}
which dispatches over
\begin{minted}[fontsize=\footnotesize]{smalltalk}
Object>>#consedObject: car
    ^ Cons car: car asCons cdr: self asCons

BlockClosure>>#consedObject: car
    ^ Chain item: car linker: self

Chain>>#consedObject: car
    ^ car cons: [self]
\end{minted}
where (i)~\Verb|Cons| objects mimic \textit{Lisp's cons cells}
\begin{minted}[fontsize=\footnotesize]{smalltalk}
Cons class>>#car: anObj cdr: anotherObj
    ^ self new
        car: anObj;
        cdr: anotherObj;
        yourself

Object>>#asCons
    ^ Cons fromObject: self

Cons class>>#fromObject: anObj
    ^ anObj
\end{minted}
and (ii)~the abstract class \Verb|Chain| provides the following class-side messages
\begin{minted}[fontsize=\footnotesize]{smalltalk}
Chain class>>#bottom
    ^ Bottom new

Chain class>>#item: anObj linker: aBlockClosure
    ^ Knot new
        item: anObj;
        linker: aBlockClosure;
        yourself

Chain class>>#with: anObj
    ^ anObj cons: Chain bottom

Chain class>>#repeat: anObj
    ^ anObj cons: [self repeat: anObj]
\end{minted}
having two subclasses \Verb|Knot| and \Verb|Bottom|. A comprehensive
test case follows
\begin{minted}[fontsize=\footnotesize]{smalltalk}
ChainTest>>#testNumbers
    self
        assert: Chain bottom contents
        equals: LinkedList new.
    self
        assert: (self nats next: 10) contents
        equals: (0 to: 9).
    self
        assert: (self zeroOneTwo next: 10) contents
        equals: (0 to: 2).
    self
        assert: (self zeroOneTwo next: 1) contents
        equals: (0 to: 0).
    self
        assert: (self fibs next: 10) contents
        equals: {0. 1. 1. 2. 3. 5. 8. 13. 21. 34}.
    self
        assert: ((self collatz: 10) next: 10) contents
        equals: {10. 5. 16. 8. 4. 2. 1. 4. 2. 1}.
    self
        assert: ((self gcd: 18 gcd: 32) next: 10) contents
        equals: {18. 32. 18. 14. 4. 2. 2}.
    self
        assert: ((Chain repeat: 4) next: 10) contents
        equals: {4. 4. 4. 4. 4. 4. 4. 4. 4. 4}
\end{minted}

On one hand, when a \Verb|Chain| object (i)~receives the selector
\Verb|#next:| that carries a \Verb|SmallInteger| object $n$ then it keeps at most
$n$ objects and then truncate itself, precisely
\begin{minted}[fontsize=\footnotesize]{smalltalk}
Bottom>>#next: n
    ^ self

Knot>>#next: n
    ^ n = 1
        ifTrue: [Chain with: item]
        ifFalse: [item cons: [self next next: n - 1]]
\end{minted}

On the other hand, when a \Verb|Chain| object (ii)~receives the selector
\Verb|#contents| it builds a \Verb|LinkedList| object \textit{eagerly},
\begin{minted}[fontsize=\footnotesize]{smalltalk}
Bottom>>#contents
    ^ LinkedList new

Knot>>#contents
    ^ self next contents
        addFirst: item;
        yourself
\end{minted}
where \Verb|Knot| objects provide the following simple but fundamental message
\begin{minted}[fontsize=\footnotesize]{smalltalk}
Knot>>#next
    ^ linker value
\end{minted}
Observe that the \Verb|Chain| hierarchy support the protocol provided
by \Verb|ReadStream| objects because of the messages \Verb|#next|, \Verb|#next:|
and \Verb|#contents|.

\subsection{\texttt{Runner}s, \texttt{Reifier}s and \texttt{Var}s}

Recall that a \Verb|Goal| object responds with a \Verb|Chain| that contains either zero,
one, or more substitutions when it receives the message \Verb|#onState:| and it
delegates the task to process those substitutions to \Verb|Runner| objects,
\begin{minted}[fontsize=\footnotesize]{smalltalk}
Goal>>#solutions
    ^ Runner onGoal: self

Runner class>>#onGoal: aGoal
    ^ self new
        goal: aGoal;
        yourself
\end{minted}
which are able to process both (i)~all of them with the message
\begin{minted}[fontsize=\footnotesize]{smalltalk}
Runner>>#all
    ^ self run contents
\end{minted}
and (ii)~at most $n$ of them with the message
\begin{minted}[fontsize=\footnotesize]{smalltalk}
Runner>>#atMost: anInteger
    ^ (self run next: anInteger) contents
\end{minted}
relying on the unary message
\begin{minted}[fontsize=\footnotesize]{smalltalk}
Runner>>#run
    | states vars selector |
    states := goal onState: Dictionary new asState.
    vars := goal vars ifEmpty: [LinkedList with: Var tautology].
    selector := vars size = 1 ifTrue: #first ifFalse: #yourself.
    ^ states
        collect: [:s | selector
                        value: (s reifier reifyVars: vars)]
\end{minted}
having the responsibilities
\begin{itemize}

\item to send the message \Verb|#onState:| to \Verb|Goal| objects to start the
resolution process from an empty state, this can be considered the entry point of
the ``computation'' that produces a \Verb|Chain| of substitutions;

\item to take into account the variables with respect to results should be
provided; as a particular case, if the root goal doesn't introduce any logic
variable, by means of the auxiliary messages
\begin{minted}[fontsize=\footnotesize]{smalltalk}
Goal>>#vars
    ^ {}

Fresh>>#vars
    ^ vars
\end{minted}
then it is equivalent to prove a \textit{tautology}, represented by
\begin{minted}[fontsize=\footnotesize]{smalltalk}
Var class>>#tautology
    ^ #tautology
\end{minted}
which require to decide that a relation is always true regardless values taken
by logic variables;

\item to simplify the results presentation when the root \Verb|Goal| object
introduces \textit{exactly one} logic variable: in this case a
\Verb|LinkedList| of values for that variable will be produced; otherwise, a
\Verb|LinkedList| of \Verb|LinkedList| of values will be produces each one for
each tuple of logic variables introduced by the root \Verb|Goal| object -- this
point will be clear when we'll explain \Verb|Fresh| goals that are the only
ones able to introduce logic variables;

\item to map each substitution (always translate to \Verb|State| object) that
satisfies the root \Verb|Goal| object do \textit{reify} the required variables,
this is accomplished by the messages
\begin{minted}[fontsize=\footnotesize]{smalltalk}
s reifier reifyVars: vars
\end{minted}
within the block argument of the returning \Verb|#collect:| message send.
\end{itemize}
The last point shows that \Verb|Chain| objects are \textit{functors} according
to the Haskell terminology because they answer to the message \Verb|#collect:|,
which is a synonism for the function \Verb|fmap :: (a -> b) -> f a -> f b|, as
\begin{minted}[fontsize=\footnotesize]{smalltalk}
Chain>>#collect: aBlockClosure
    ^ self

Knot>>#collect: aBlockClosure
    | v |
    v := aBlockClosure value: item.
    ^ v cons: [self next collect: aBlockClosure]
\end{minted}
respectively.  We are now ready to describe the reification process which
requires
\begin{minted}[fontsize=\footnotesize]{smalltalk}
State>>#reifier
    ^ Reifier on: self

Reifier>>#reifyVars: vars
    | walker reifier |
    walker := state walker.
    reifier := Dictionary new asState reifier.
    ^ vars
        collect: [:v |
                    | w s |
                    w := walker walkStar: v.
                    s := reifier reify: w.
                    s walker walkStar: w]
\end{minted}
which uses the message
\begin{minted}[fontsize=\footnotesize]{smalltalk}
Reifier>>#reify: anObj 
    ^ (state walk: anObj) 
        reifyUsing: self
\end{minted}
to dispatch the message \Verb|#reifyUsing:| over
\begin{minted}[fontsize=\footnotesize]{smalltalk}
Object>>#reifyUsing: aReifier
    ^ aReifier reifyObject: self 

Var>>#reifyUsing: aReifier 
    ^ aReifier reifyVar: self 

ReifiedVar>>#reifyUsing: aReifier 
    ^ aReifier reifyReifiedVar: self

Array>>#reifyUsing: aReifier
    ^ aReifier reifyArray: self 

LinkedList>>#reifyUsing: aReifier 
    ^ aReifier reifyLinkedList: self 

Cons>>#reifyUsing: aReifier 
    ^ aReifier reifyCons: self 
\end{minted}
and to collect replies as
\begin{minted}[fontsize=\footnotesize]{smalltalk}
Reifier>>#reifyObject: anObj 
    ^ state

Reifier>>#reifyVar: aVar 
    ^ state reifyVar: aVar

Reifier>>#reifyReifiedVar: aVar 
    ^ state

Reifier>>#reifyArray: aCollection 
    ^ self reifyCollection: aCollection

Reifier>>#reifyArray: aCollection 
    ^ self reifyCollection: aCollection

Reifier>>#reifyArray: aCollection 
    ^ self reifyCollection: aCollection

Reifier>>#reifyCollection: aCollection 
    ^ aCollection inject: state into: [:s :c | s reifier reify: c]
\end{minted}
requiring the auxiliary message
\begin{minted}[fontsize=\footnotesize]{smalltalk}
State>>#reifyVar: aVar 
    substitution
        at: aVar
        ifAbsentPut: [ReifiedVar new 
                        id: substitution size; 
                        yourself]
\end{minted}
and
\begin{minted}[fontsize=\footnotesize]{smalltalk}
State>>#walk: anObj 
    | k |
    k := anObj.
    [k := substitution at: k ifAbsent: [^ k]] repeat
\end{minted}

\hrulefill

where a \Verb|Walker| object is built by
\begin{minted}[fontsize=\footnotesize]{smalltalk}
Walker class>>#on: aState
    ^ self new
        state: aState;
        yourself
\end{minted}
and it responds to the selector \Verb|#walkStar| with
\begin{minted}[fontsize=\footnotesize]{smalltalk}
Walker>>#walkStar: aVar
    ^ (state walk: aVar)
        walkStarUsingWalker: self
\end{minted}
which relies on
\begin{minted}[fontsize=\footnotesize]{smalltalk}
State>>#walk: anObj
    | k |
    k := anObj.
    [k := substitution
            at: k
            ifAbsent: [^ k]] repeat
\end{minted}
and objects of different classes responds to \Verb|#walkStarUsingWalker:| as follows
\begin{minted}[fontsize=\footnotesize]{smalltalk}
Object>>#walkStarUsingWalker: aWalker
    ^ aWalker walkStarObject: self

Array>>#walkStarUsingWalker: aWalker
    ^ aWalker walkStarArray: self

LinkedList>>#walkStarUsingWalker: aWalker
    ^ aWalker walkStarLinkedList: self

Cons>>#walkStarUsingWalker: aWalker
    ^ aWalker walkStarCons: self
\end{minted}
dispatching over \Verb|Walker| objects
\begin{minted}[fontsize=\footnotesize]{smalltalk}
Walker>>#walkStarObject: anObj
    ^ state walk: anObj

Walker>>#walkStarArray: aCollection
    ^ aCollection
        collect: [:each | self walkStar: each]

Walker>>#walkStarLinkedList: aCollection
    ^ aCollection
        collect: [:each | self walkStar: each]

Walker>>#walkStarCons: aCons
    | car cdr |
    car := self walkStar: aCons car.
    cdr := self walkStar: aCons cdr.
    ^ car cons: cdr
\end{minted}
respectively.

\section{Bootstrapping $\mu$Kanren}

We believe that an unusual development process can be done in Smalltalk and it
is the one that had driven and currently drives the present description.
Instead of creating classes first, we start by looking at some \textit{existing
object} that can be a good candidate to take into account our messages; in
sight of this approach, we give the fundamentals of $\mu$Kanren -- more
generally, for a class of logic programming languages -- starting from the two
nice boolean objects.

\subsection{\texttt{Succeed} and \texttt{Fail} classes}

Let us consider the object \Verb|false| whose responsibility is to represent
\textit{falsehood}, hence it should be a good candidate to provide us something
related to this from the \textit{relational} point of view,
\begin{minted}[fontsize=\footnotesize]{smalltalk}
false asGoal
\end{minted}
namely, \Verb|false| can build for us an object that represents a relation that
cannot be satisfied at all responding to the message selector \Verb|asGoal| --
don't be scared by multiple names, we freely interchange the terms
\textit{goal} and \textit{relation} postponing to later sections the
description of their technical difference. Moreover, let us
assume that the infrastructure that actually builds results is given and implemented
by the block \texttt{B := [:g | g solutions all]}, we write our first test case:
\begin{minted}[fontsize=\footnotesize]{smalltalk}
GoalTest>>#testFail
    self
        assert: false asGoal solutions all
        equals: {}
\end{minted}
Now we extend the \Verb|False| class with the following \Verb|asGoal| message
definition
\begin{minted}[fontsize=\footnotesize]{smalltalk}
False>>#asGoal
    ^ Goal fail
\end{minted}
and of the message
\begin{minted}[fontsize=\footnotesize]{smalltalk}
Goal>>#fail
    ^ Fail new
\end{minted}
such that to say that \verb|Fail| denotes the impossibility to satisfy a relation
is the same to code the message
\begin{minted}[fontsize=\footnotesize]{smalltalk}
Fail>>#onState: aState
    ^ Chain bottom
\end{minted}
which will be used within the message \Verb|solutions| sent in the block
\Verb|B| -- a lot remains to be explained, such as the \Verb|Chain| class and
streams in particular, all in what follows.

In parallel, we reproduce for the object \Verb|true| the steps did for
the object \Verb|false|; first,
\begin{minted}[fontsize=\footnotesize]{smalltalk}
GoalTest>>#testSucceed
    self
        assert: true asGoal solutions all
        equals: {Var tautology}
\end{minted}
second,
\begin{minted}[fontsize=\footnotesize]{smalltalk}
True>>#asGoal
    ^ Goal succeed
\end{minted}
third,
\begin{minted}[fontsize=\footnotesize]{smalltalk}
Goal>>#succeed
    ^ Succeed new
\end{minted}
fourth,
\begin{minted}[fontsize=\footnotesize]{smalltalk}
Succeed>>#onState: aState
    ^ Chain with: aState
\end{minted}

\section{Theoretical backgrounds}

We quickly review some mathematical backgrounds underlying logic systems;
in particular deductions are performed according to \emph{resolution} methods.

\subsection{Resolution by refutation}

Let $\alpha$ be a sentence in \emph{conjunctive normal
form} (CNF for short) and $M(\alpha)$ the set of models that satisfy $\alpha$
-- recall that a \emph{model} is a set of assignments of values to free
variables in $\alpha$ that make the whole sentence true.  We say that, on one
hand, $\alpha$ is \textit{valid} if it is true in \textit{all} models; on the
other hand, $\alpha$ is \textit{satisfiable} if it is true in \textit{some}
model.

Let $\models$ and $\Rightarrow$ denote the \textit{entail} and \textit{imply}
relations, respectively; so,
\begin{displaymath}
\begin{split}
\alpha \models \beta \leftrightarrow
M(\alpha) \subseteq M(\beta) \leftrightarrow
(\alpha \Rightarrow \beta) \text{ is valid } \leftrightarrow
\neg(\neg\alpha \vee \beta) \text{ is unsatisfiable.}
\end{split}
\end{displaymath}
Therefore, to prove a sentence $\alpha$ reduces to prove
$\neg\alpha\models\perp$, where $\perp$ denotes the empty clause $()$, namely
\textit{falsehood}, which can be deduced by the \textit{resolution rule}
introduced in \citep{Robinson:1965:MLB:321250.321253}; if $l_{i}$ and $m_{r}$
bound to the same variable then
\begin{displaymath}
{\left(l_{0}, \ldots, l_{i}, \ldots, l_{j-1}\right) \quad \left(m_{0}, \ldots, m_{r},\ldots, m_{k-1}\right) \quad l_{i} = \neg m_{r}
\over
\left(l_{0},\ldots, l_{i-1},l_{i+1}, \ldots,l_{j-1}, m_{0},\ldots, m_{r-1},m_{r+1},\ldots, m_{k-1}\right)}
\end{displaymath}
where $\left(l_{0},\ldots, l_{i}, \ldots, l_{j-1}\right) = l_{0}\vee \ldots
\vee l_{i} \vee \ldots \vee l_{j-1}$, for all $l_{q}, m_{w} \in\lbrace 0,1\rbrace$.
This rule is a \textit{complete} inference algorithm, namely it enumerates
\emph{all} deductions for a given sentence, although requiring
\emph{exponential} time; for the sake of concreteness, the \textit{DPLL}
algorithm \citep{Davis:1962:MPT:368273.368557}
is a recursive, depth-first enumeration of models using the
resolution rule paired with heuristics \textit{early termination},
\textit{pure symbol} and \textit{unit clause} to speed up.

\subsection{Resolution by unification}

This method was introduced in \citep{robinson_unif} and
improved by \citep{Martelli:1982:EUA:357162.357169}; it consists of solving
\textit{equations among symbolic expressions}. A \textit{solution} is denoted
as a \textit{substitution} $\theta$, namely a mapping that assigns a symbolic
values to free variables; let $G$ be a set of equations, unification proceeds
according to the following rules:
\begin{description}
\item[delete] $G \cup \lbrace t = t \rbrace \rightarrow G$
\item[decompose] $G \cup \lbrace f(s_{0}, \ldots, s_{k}) = f(t_{0}, \ldots, t_{k})\rbrace$ entails
$$G \cup \lbrace s_{0}=t_{0},\ldots, s_{k}=t_{k} \rbrace$$
\item[conflict] if $f\neq g \vee k\neq m$ then $$G \cup \lbrace f(s_{0}, \ldots, s_{k}) = g(t_{0}, \ldots, t_{m})\rbrace \rightarrow \,\perp$$
\item[eliminate] if $x \not\in vars(t)$ and $x \in vars(G)$ then $$G \cup \lbrace x = t\rbrace \rightarrow G\lbrace x \mapsto t\rbrace \cup \left\lbrace x \triangleq t\right\rbrace $$
\item[occur check] if $x \in vars(f(s_{0},\ldots,s_{k}))$ then $$G \cup \lbrace x = f(s_{0}, \ldots, s_{k})\rbrace \rightarrow \,\perp;$$
without it, generating a $\theta$ is a
\emph{recursive enumerable} problem.
\end{description}

\begin{example}
Let $x$ and $y$ be free variables, the set
$$\lbrace cons(x,cons(x,nil)) = cons(2,y)\rbrace$$
has solution $\theta = \lbrace x \mapsto 2, y \mapsto cons(2,nil) \rbrace$;
on the contrary, the set
$$ \lbrace y = cons(2,y) \rbrace $$
has no \textit{finite} solution, nonetheless
$$\theta = \lbrace y \mapsto cons(2,cons(2,cons(2,...))) \rbrace$$
is a solution upto \textit{bisimulation}
\citep{10.1007/BFb0017309, DBLP:books/daglib/0067019}.
\end{example}
\section{Conclusions and further works}


\begin{itemize}
\item A quine generator is described in \citep{Byrd:2012:MLU:2661103.2661105}
and we aim to reproduce it.

\item using \emph{OCaml} the syntax of the language can be cleanly defined by
algebraic datatypes and the type system helps in the design of the required
operators.  While object-oriented languages make extensibility a relative
simple task, this implementation is harder unless we can make changes to basic
types, at least for what concerns the central \emph{unification} procedure.
All of this is preparatory to modify the tactics mechanism of the \emph{HOL
Light} theorem prover in order to introduce the relational paradigm in that
environment.
\end{itemize}

\bibliographystyle{plainnat}
\bibliography{biblio}
\end{document}
