
\documentclass[a4paper,12pt]{article}

\usepackage{inputenc}
\usepackage{euler}
\usepackage[T1]{fontenc}
\usepackage{libertine}
\usepackage{lipsum}
\usepackage{fancyvrb}
\usepackage{url}
\usepackage[english]{babel}
\usepackage{amsmath}
\usepackage{amsthm}
\usepackage{amssymb}
\usepackage{acronym}
\usepackage{hyperref}
\usepackage{tabu}
\usepackage{rotating}
\usepackage{mathdots}
\usepackage{minted}
\usepackage{units}
\usepackage{float}
\usepackage{bbold}
\usepackage[sort&compress,square,comma,authoryear]{natbib}

\fvset{fontsize=\normalsize}
\setmonofont[Scale=0.8]{Menlo}
\usemintedstyle{xcode}
\hypersetup{colorlinks=false, citecolor=Violet}

\newtheorem{theorem}{Theorem}
\newtheorem{lemma}[theorem]{Lemma}
\newtheorem{proposition}[theorem]{Proposition}
\newtheorem{corollary}[theorem]{Corollary}
\newtheorem{definition}[theorem]{Definition}
\newtheorem{remark}[theorem]{Remark}
\newtheorem{example}[theorem]{Example}

\author{Massimo Nocentini}
\title{Relational programming in Smalltalk}

\begin{document}

\maketitle

\begin{abstract}
This booklet is an extended description of $mu$Kanren, a relational interpreter
formerly implemented in Scheme and ported in Smalltalk for the sake of
understanding and, of course, fun. It enjoy simplicity and elegance despite the
complex nature of logic systems; moreover, using an uniform but powerful
language such as Smalltalk it is possible to appreciate and benefit from them
all.
\end{abstract}

\section{Introduction}

I believe \textit{microkanren} is, first of all, an \textit{educational beast},
concerning unification, lazy streams, backtracking and optimization; the
abstract definition was shown in \citep{Hemann:muKanren}.
I repeat the exercise of writing it in \citep{Nocentini:kanrens}:
\begin{itemize}
\item Python, native \verb|generator|s :) limits on recursive calls :(
\item OCaml, algebraic datatypes :) hard to extend :(
\item Smalltalk, simple, fast and clear :) many dispatching msgs :/
\end{itemize}

\begin{example}
In math a relation $P$ is usually characterized by
\begin{displaymath}
\forall a,b,c.\,P(a,b,c) \leftrightarrow a + b = c \quad\text{entails}\quad P(1,2,3)
\end{displaymath}
can be expressed using either the \textit{imperative style}
\begin{minted}[fontsize=\footnotesize]{smalltalk}
a := 1.  b := 2.  c := a + b.  Object assert: [ c = 3 ]
\end{minted}
or the \textit{functional style}
\begin{minted}[fontsize=\footnotesize]{smalltalk}
Object assert: [ ([ :a :b | a + b ] value: 1 value: 2) = 3 ]
\end{minted}
or, finally, the \textit{declarative style}
\begin{minted}[fontsize=\footnotesize]{smalltalk}
Object assert: [ [ :a :b :c | a + b = c ] value: 1 value: 2 value: 3 ]
\end{minted}
which closely resembles the mathematical statement.
\end{example}

We quickly review some mathematical backgrounds underlying logic systems;
in particular deductions are performed according to \emph{resolutions}

\begin{description}

\item[by refutation] Let $\alpha$ be a sentence in \emph{conjunctive normal
form} (CNF for short) and $M(\alpha)$ the set of models that satisfy $\alpha$
-- recall that a \emph{model} is a set of assignments of values to free
variables in $\alpha$ that make the whole sentence true.  We say that, on one
hand, $\alpha$ is \textit{valid} if it is true in \textit{all} models; on the
other hand, $\alpha$ is \textit{satisfiable} if it is true in \textit{some}
model.

Let $\models$ and $\Rightarrow$ denote the \textit{entail} and \textit{imply}
relations, respectively; so,
\begin{displaymath}
\begin{split}
\alpha \models \beta \leftrightarrow
M(\alpha) \subseteq M(\beta) \leftrightarrow
(\alpha \Rightarrow \beta) \text{ is valid } \leftrightarrow
\neg(\neg\alpha \vee \beta) \text{ is unsatisfiable.}
\end{split}
\end{displaymath}
Therefore, to prove a sentence $\alpha$ reduces to prove
$\neg\alpha\models\perp$, where $\perp$ denotes the empty clause $()$, namely
\textit{falsehood}, which can be deduced by the \textit{resolution rule}
introduced in \citep{Robinson:1965:MLB:321250.321253}; if $l_{i}$ and $m_{r}$
bound to the same variable then
\begin{displaymath}
{\left(l_{0}, \ldots, l_{i}, \ldots, l_{j-1}\right) \quad \left(m_{0}, \ldots, m_{r},\ldots, m_{k-1}\right) \quad l_{i} = \neg m_{r}
\over
\left(l_{0},\ldots, l_{i-1},l_{i+1}, \ldots,l_{j-1}, m_{0},\ldots, m_{r-1},m_{r+1},\ldots, m_{k-1}\right)}
\end{displaymath}
where $\left(l_{0},\ldots, l_{i}, \ldots, l_{j-1}\right) = l_{0}\vee \ldots
\vee l_{i} \vee \ldots \vee l_{j-1}$, for all $l_{q}, m_{w} \in\lbrace 0,1\rbrace$.
This rule is a \textit{complete} inference algorithm, namely it enumerates
\emph{all} deductions for a given sentence, although requiring
\emph{exponential} time; for the sake of concreteness, the \textit{DPLL}
algorithm \citep{Davis:1962:MPT:368273.368557}
is a recursive, depth-first enumeration of models using the
resolution rule paired with heuristics \textit{early termination},
\textit{pure symbol} and \textit{unit clause} to speed up.

\item[by unification] is a the process introduced in \citep{robinson_unif} and
improved by \citep{Martelli:1982:EUA:357162.357169}; it consists of solving
\textit{equations among symbolic expressions}. A \textit{solution} is denoted
as a \textit{substitution} $\theta$, namely a mapping that assigns a symbolic
values to free variables; let $G$ be a set of equations, unification proceeds
according to the following rules:
\begin{description}
\item[delete] $G \cup \lbrace t = t \rbrace \rightarrow G$
\item[decompose] $G \cup \lbrace f(s_{0}, \ldots, s_{k}) = f(t_{0}, \ldots, t_{k})\rbrace$ entails
$$G \cup \lbrace s_{0}=t_{0},\ldots, s_{k}=t_{k} \rbrace$$
\item[conflict] if $f\neq g \vee k\neq m$ then $$G \cup \lbrace f(s_{0}, \ldots, s_{k}) = g(t_{0}, \ldots, t_{m})\rbrace \rightarrow \,\perp$$
\item[eliminate] if $x \not\in vars(t)$ and $x \in vars(G)$ then $$G \cup \lbrace x = t\rbrace \rightarrow G\lbrace x \mapsto t\rbrace \cup \left\lbrace x \triangleq t\right\rbrace $$
\item[occur check] if $x \in vars(f(s_{0},\ldots,s_{k}))$ then $$G \cup \lbrace x = f(s_{0}, \ldots, s_{k})\rbrace \rightarrow \,\perp;$$
without it, generating a $\theta$ is a
\emph{recursive enumerable} problem.
\end{description}

\end{description}

\begin{example}
Let $x$ and $y$ be free variables, the set
$$\lbrace cons(x,cons(x,nil)) = cons(2,y)\rbrace$$
has solution $\theta = \lbrace x \mapsto 2, y \mapsto cons(2,nil) \rbrace$;
on the contrary, the set
$$ \lbrace y = cons(2,y) \rbrace $$
has no \textit{finite} solution, nonetheless
$$\theta = \lbrace y \mapsto cons(2,cons(2,cons(2,...))) \rbrace$$
is a solution upto \textit{bisimulation}
\citep{10.1007/BFb0017309, DBLP:books/daglib/0067019}.
\end{example}


\section{Conclusions}

A quine generator is described in \citep{Byrd:2012:MLU:2661103.2661105}.

\bibliographystyle{plainnat}
\bibliography{biblio}
\end{document}
