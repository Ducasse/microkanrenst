
\documentclass[a4paper,12pt]{article}

\usepackage{inputenc}
\usepackage{euler}
\usepackage[T1]{fontenc}
\usepackage{libertine}
\usepackage{lipsum}
\usepackage{fancyvrb}
\usepackage{url}
\usepackage[english]{babel}
\usepackage{amsmath}
\usepackage{amsthm}
\usepackage{amssymb}
\usepackage{acronym}
\usepackage{hyperref}
\usepackage{tabu}
\usepackage{rotating}
\usepackage{mathdots}
\usepackage{minted}
\usepackage{units}
\usepackage{float}
\usepackage{bbold}
\usepackage[sort&compress,square,comma,authoryear]{natbib}

\fvset{fontsize=\normalsize}
\setmonofont[Scale=0.8]{Menlo}
\usemintedstyle{xcode}
\hypersetup{colorlinks=false, citecolor=Violet}

\newtheorem{theorem}{Theorem}
\newtheorem{lemma}[theorem]{Lemma}
\newtheorem{proposition}[theorem]{Proposition}
\newtheorem{corollary}[theorem]{Corollary}
\newtheorem{definition}[theorem]{Definition}
\newtheorem{remark}[theorem]{Remark}
\newtheorem{example}[theorem]{Example}

\newcommand{\stcode}[1]{\begin{minted}[fontsize=\footnotesize]{smalltalk} #1 \end{minted}}

\author{Massimo Nocentini}
\title{Relational programming in Smalltalk}

\begin{document}

\maketitle

\begin{abstract}
This booklet is an extended description of $\mu$Kanren, a relational
interpreter formerly implemented in Scheme and ported in Smalltalk for the sake
of understanding and, of course, fun. It enjoys simplicity and elegance despite
the complex nature of logic systems; moreover, using an uniform but powerful
language such as Smalltalk it is possible to appreciate and benefit from both
of them.
\end{abstract}

\section*{Introduction}

This document extends the presentation of $\mu$Kanren shown at the last ESUG
conference \citep{ESUG2018} in two ways: (i)~it recaps existing ``literature''
on the subject and (ii)~it describes in more details our implementation coded
in Smalltalk.

It is a waste of time to repeat ourself on $\mu$Kanren's definition; on the
contrary, we believe interesting to look up and to watch over the many
implementations it has inspired since its seminal paper \citep{Hemann:muKanren}
and how programmers stretch the canonical implementation to take advantage of
their favorite programming language's features.

Certainly the book \citep{Friedman:Reasoned:Schemer} is the main reference
written in the same style of the other \emph{little books} of which Dan
Friedman coauthors them all; moreover, Byrd's dissertation \citep{Byrd:PhD}
shows advanced stuffs and techniques while the repository
\url{http://minikanren.org/} -- updated regularly by Byrd himself -- keeps
pointers to implementations, articles and presentations. All these references
use the Scheme programming language, $\mu$Kanren is \emph{purely functional}
therefore any language that has the same power of the $\lambda$-calculus is a
candidate for implementing it, Scheme in particular; in parallel, the
implementation with symbolic constraints has the same nature but uses Scheme
macrology and in some aspect it requires little effort to port to other
languages that haven't macros.

Another implementation using the \emph{Clojure} dialect of Lisp is
\citep{clojure:core.logic} which targets the JVM and provides constraints and
nominal logic programming; on the other hand, \citep{sullivan:microkanrenhs}
is a clean and concise implementation using Haskell, while many others simpler
versions can be found, written for the sake of understanding and, ultimately,
for fun.

Finally, your humble authors believe that $\mu$Kanren is a sophisticated
\textit{educational machine} because it concerns and mixes together unification,
lazy streams, backtracking and optimization techniques; moreover, it allows
programmers to write their own version according to their personal taste and
style, ranging from implementations pretty imperative to ones built over
categories.  We repeat the exercise of writing it in \citep{Nocentini:kanrens}
targeting different languages, in particular
\begin{itemize}
\item using \emph{Python} we take advantage of native \verb|generator|s objects
and we provide a \emph{fair} enumeration strategy that the canonical one
hasn't; on the contrary, we experience computational limits because of
\texttt{RecursionError} exceptions. This implementation is complete and
parallels the one shown in the reference book, accompanied by a unit test suite
that covers \emph{all} questions explained in the book and some new
applications such as (i)~finding fixed points of a set of inference rules and
enumeration of classes of combinatorial objects.
\item using \emph{OCaml} the syntax of the language can be cleanly defined by
algebraic datatypes and the type system helps in the design of the required
operators.  While object-oriented languages make extensibility a relative
simple task, this implementation is harder unless we can make changes to basic
types, at least for what concerns the central \emph{unification} procedure.
All of this is preparatory to modify the tactics mechanism of the \emph{HOL
Light} theorem prover in order to introduce the relational paradigm in that
environment.
\end{itemize}
The current work describes a Smalltalk implementation of the basic system
with the aim to learn and experiment with a pure object environment, such as 
Pharo and Squeak. Starting small, the next example we show three different
ways to express equality between \verb|SmallInteger| objects:

\begin{example}
In math a relation $P$ is usually characterized by a statement such as
\begin{displaymath}
\forall a,b,c.\,P(a,b,c) \leftrightarrow a + b = c \quad\text{from which}\quad
P(1,2,3) \quad\text{can be proved.}
\end{displaymath}
The same check can be expressed in Smalltalk using either the
\textit{imperative style}
\begin{minted}[fontsize=\footnotesize]{smalltalk}
a := 1.  b := 2.  c := a + b.  Object assert: [ c = 3 ]
\end{minted}
or the \textit{functional style}
\begin{minted}[fontsize=\footnotesize]{smalltalk}
Object assert: [ ([ :a :b | a + b ] value: 1 value: 2) = 3 ]
\end{minted}
or, finally, the \textit{declarative style}
\begin{minted}[fontsize=\footnotesize]{smalltalk}
Object assert: [ [ :a :b :c | a + b = c ] value: 1 value: 2 value: 3 ]
\end{minted}
which closely resembles the mathematical statement.
\end{example}

\tableofcontents

\section{Bootstrapping $\mu$Kanren}

We believe that an unusual development process can be done in Smalltalk and it
is the one that had driven and currently drives the present description.
Instead of creating classes first, we start by looking at some \textit{existing
object} that can be a good candidate to take into account our messages; in
sight of this approach, we give the fundamentals of $\mu$Kanren -- more
generally, for a class of logic programming languages -- starting from the two
nice boolean objects. 
    
\subsection{\texttt{Succeed} and \texttt{Fail} classes}    

Let us consider the object \Verb|false| whose responsibility is to represent
\textit{falsehood}, hence it should be a good candidate to provide us something
related to this from the \textit{relational} point of view,
\begin{minted}[fontsize=\footnotesize]{smalltalk}
false asGoal
\end{minted}
namely, \Verb|false| can build for us an object that represents a relation that
cannot be satisfied at all responding to the message selector \Verb|asGoal| --
don't be scared by multiple names, we freely interchange the terms
\textit{goal} and \textit{relation} postponing to later sections the
description of their technical difference. Moreover, let us 
assume that the infrastructure that actually builds results is given and implemented
by the block \texttt{B := [:g | g solutions all]}, we write our first test case:
\begin{minted}[fontsize=\footnotesize]{smalltalk}
GoalTest>>#testFail
    self 
        assert: false asGoal solutions all 
        equals: {}
\end{minted}
Now we extend the \Verb|False| class with the following \Verb|asGoal| message
definition
\begin{minted}[fontsize=\footnotesize]{smalltalk}
False>>#asGoal
    ^ Goal fail
\end{minted}
and of the message
\begin{minted}[fontsize=\footnotesize]{smalltalk}
Goal>>#fail
    ^ Fail new
\end{minted}
such that to say that \verb|Fail| denotes the impossibility to satisfy a relation
is the same to code the message
\begin{minted}[fontsize=\footnotesize]{smalltalk}
Fail>>#onState: aState 
    ^ Chain bottom
\end{minted}
which will be used within the message \Verb|solutions| sent in the block
\Verb|B| -- a lot remains to be explained, such as the \Verb|Chain| class and
streams in particular, all in what follows.

In parallel, we reproduce for the object \Verb|true| the steps did for
the object \Verb|false|; first,
\begin{minted}[fontsize=\footnotesize]{smalltalk}
GoalTest>>#testSucceed
    self 
        assert: true asGoal solutions all 
        equals: {Var tautology}
\end{minted}
second,
\begin{minted}[fontsize=\footnotesize]{smalltalk}
True>>#asGoal
    ^ Goal succeed
\end{minted}
third,
\begin{minted}[fontsize=\footnotesize]{smalltalk}
Goal>>#succeed
    ^ Succeed new
\end{minted}
fourth,
\begin{minted}[fontsize=\footnotesize]{smalltalk}
Succeed>>#onState: aState 
    ^ Chain with: aState
\end{minted}

\section{Theoretical backgrounds}

We quickly review some mathematical backgrounds underlying logic systems;
in particular deductions are performed according to \emph{resolution} methods.

\subsection{Resolution by refutation} 

Let $\alpha$ be a sentence in \emph{conjunctive normal
form} (CNF for short) and $M(\alpha)$ the set of models that satisfy $\alpha$
-- recall that a \emph{model} is a set of assignments of values to free
variables in $\alpha$ that make the whole sentence true.  We say that, on one
hand, $\alpha$ is \textit{valid} if it is true in \textit{all} models; on the
other hand, $\alpha$ is \textit{satisfiable} if it is true in \textit{some}
model.

Let $\models$ and $\Rightarrow$ denote the \textit{entail} and \textit{imply}
relations, respectively; so,
\begin{displaymath}
\begin{split}
\alpha \models \beta \leftrightarrow
M(\alpha) \subseteq M(\beta) \leftrightarrow
(\alpha \Rightarrow \beta) \text{ is valid } \leftrightarrow
\neg(\neg\alpha \vee \beta) \text{ is unsatisfiable.}
\end{split}
\end{displaymath}
Therefore, to prove a sentence $\alpha$ reduces to prove
$\neg\alpha\models\perp$, where $\perp$ denotes the empty clause $()$, namely
\textit{falsehood}, which can be deduced by the \textit{resolution rule}
introduced in \citep{Robinson:1965:MLB:321250.321253}; if $l_{i}$ and $m_{r}$
bound to the same variable then
\begin{displaymath}
{\left(l_{0}, \ldots, l_{i}, \ldots, l_{j-1}\right) \quad \left(m_{0}, \ldots, m_{r},\ldots, m_{k-1}\right) \quad l_{i} = \neg m_{r}
\over
\left(l_{0},\ldots, l_{i-1},l_{i+1}, \ldots,l_{j-1}, m_{0},\ldots, m_{r-1},m_{r+1},\ldots, m_{k-1}\right)}
\end{displaymath}
where $\left(l_{0},\ldots, l_{i}, \ldots, l_{j-1}\right) = l_{0}\vee \ldots
\vee l_{i} \vee \ldots \vee l_{j-1}$, for all $l_{q}, m_{w} \in\lbrace 0,1\rbrace$.
This rule is a \textit{complete} inference algorithm, namely it enumerates
\emph{all} deductions for a given sentence, although requiring
\emph{exponential} time; for the sake of concreteness, the \textit{DPLL}
algorithm \citep{Davis:1962:MPT:368273.368557}
is a recursive, depth-first enumeration of models using the
resolution rule paired with heuristics \textit{early termination},
\textit{pure symbol} and \textit{unit clause} to speed up.

\subsection{Resolution by unification} 

This method was introduced in \citep{robinson_unif} and
improved by \citep{Martelli:1982:EUA:357162.357169}; it consists of solving
\textit{equations among symbolic expressions}. A \textit{solution} is denoted
as a \textit{substitution} $\theta$, namely a mapping that assigns a symbolic
values to free variables; let $G$ be a set of equations, unification proceeds
according to the following rules:
\begin{description}
\item[delete] $G \cup \lbrace t = t \rbrace \rightarrow G$
\item[decompose] $G \cup \lbrace f(s_{0}, \ldots, s_{k}) = f(t_{0}, \ldots, t_{k})\rbrace$ entails
$$G \cup \lbrace s_{0}=t_{0},\ldots, s_{k}=t_{k} \rbrace$$
\item[conflict] if $f\neq g \vee k\neq m$ then $$G \cup \lbrace f(s_{0}, \ldots, s_{k}) = g(t_{0}, \ldots, t_{m})\rbrace \rightarrow \,\perp$$
\item[eliminate] if $x \not\in vars(t)$ and $x \in vars(G)$ then $$G \cup \lbrace x = t\rbrace \rightarrow G\lbrace x \mapsto t\rbrace \cup \left\lbrace x \triangleq t\right\rbrace $$
\item[occur check] if $x \in vars(f(s_{0},\ldots,s_{k}))$ then $$G \cup \lbrace x = f(s_{0}, \ldots, s_{k})\rbrace \rightarrow \,\perp;$$
without it, generating a $\theta$ is a
\emph{recursive enumerable} problem.
\end{description}

\begin{example}
Let $x$ and $y$ be free variables, the set
$$\lbrace cons(x,cons(x,nil)) = cons(2,y)\rbrace$$
has solution $\theta = \lbrace x \mapsto 2, y \mapsto cons(2,nil) \rbrace$;
on the contrary, the set
$$ \lbrace y = cons(2,y) \rbrace $$
has no \textit{finite} solution, nonetheless
$$\theta = \lbrace y \mapsto cons(2,cons(2,cons(2,...))) \rbrace$$
is a solution upto \textit{bisimulation}
\citep{10.1007/BFb0017309, DBLP:books/daglib/0067019}.
\end{example}
\section{Conclusions}

A quine generator is described in \citep{Byrd:2012:MLU:2661103.2661105} and
we aim to reproduce it.

\bibliographystyle{plainnat}
\bibliography{biblio}
\end{document}
